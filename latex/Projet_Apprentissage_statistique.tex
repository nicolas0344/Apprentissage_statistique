%%% packages %%%
%%%%%%%%%%%%%%%%%%%%%%%%%%%%%%%%%%%%%%%%%%%%%%%%%%%%%%%%%%%%%%%%%%%%%%%%%%%%%%%  
\documentclass[frenchb]{report}
%\usepackage{natbib}
\usepackage[toc,page]{appendix}
\usepackage[dvipsnames]{xcolor}
\usepackage[french]{babel}
\usepackage{url}
\usepackage[utf8x]{inputenc}
\usepackage{graphicx}
\graphicspath{{images/}}
\usepackage{parskip}
\usepackage{fancyhdr}
\usepackage{fancyvrb}
\usepackage{vmargin}
\usepackage{xcolor}
\usepackage{bbm}
\usepackage{amsmath,amssymb}
\usepackage{amsthm}
\usepackage{dsfont}
\usepackage{stmaryrd}
\usepackage{systeme}
\usepackage{enumitem}
\usepackage{xcolor}
\usepackage{pifont}
\usepackage{textcomp}
\usepackage{calrsfs}
\usepackage[T1]{fontenc}
\usepackage[toc,page]{appendix}
\usepackage{lipsum}
\usepackage{verbatim}
\usepackage{listings}
\usepackage{adforn}
\usepackage{float}
\usepackage{subfig}

\makeatletter
\let\thetitle\@title
\let\theauthor\@author
\let\thedate\@date
\makeatother

%%% commandes mise en page %%%
%%%%%%%%%%%%%%%%%%%%%%%%%%%%%%%%%%%%%%%%%%%%%%%%%%%%%%%%%%%%%%%%%%%%%%%%%%%%%%%        
\newcommand{\ld}{\log_{2}}
\newcommand{\R}{\mathbbm{R}}
\newcommand{\N}{\mathbbm{N}}
\newcommand{\1}{\mathbbm{1}}
\newcommand{\E}{\mathbbm{E}}
\newcommand{\V}{\mathbbm{V}}
\newcommand{\prob}{\mathbbm{P}}
\newcommand{\Nc}{\mathcal{N}}
\newcommand{\Cc}{\mathcal{C}}
\newcommand{\K}{\mathcal{K}}
\newcommand{\Xti}{\widetilde{X_i}}
\newcommand{\Xtj}{\widetilde{X_j}}
\newcommand{\Xn}{\overline{X_n}}
\newcommand{\gn}{\hat{g_n}}
\newcommand{\n}{\mathcal{N}}
\newcommand{\lv}{\mathcal{L}}
\newcommand{\thetat}{\tilde{\theta}}

\newcommand{\console}[1]{\colorbox{black}{\begin{minipage}[c]{1\linewidth}\textcolor{white}{\texttt{#1}}\end{minipage}}}

\newtheorem{prop}{Proposition}
\newtheorem{thm}{Théorème}
\newtheorem{cor}{Corollaire}
\newtheorem{lem}{Lemme}
\newtheorem{hyp}{Hypothèse}
\theoremstyle{definition}\newtheorem{defn}{Définition}
\theoremstyle{definition}\newtheorem{exm}{Exemple}
\theoremstyle{definition}\newtheorem{nota}{Notation}
\theoremstyle{definition}\newtheorem{rem}{Remarque}

\renewcommand{\qedsymbol}{\adfhangingflatleafright}

\begin{document}
%%% Pour l'annexe
\def\appendixpage{\vspace*{8cm}
\begin{center}
\Huge\textbf{Annexes}
\end{center}
}
\def\appendixname{Annexe}%

\begin{titlepage}
\begin{center}
\includegraphics[scale=0.6]{logo.png}
\hfill
\includegraphics[scale=0.35]{fds_logo.png}\\[3cm]
\linespread{1.2}\huge {\bfseries Apprentissage Statistique }\\[0.5cm]
\linespread{1.2}\LARGE {\bfseries Détection de structures communautaites dans des réseaux}\\[1.5cm]
\linespread{1}

{\large Rédigé par\\}
{\Large \textsc{pralon} Nicolas}\\
{\Large \textsc{côme} Olivier}\\
{\Large \textsc{sene} Assane}\\[1cm]

\includegraphics[scale=0.7]{imag_logo.png}

\end{center}
\end{titlepage}
%%%%%%%%%%%%%%%%%%%%%%%%%%%%%%%%%%%%%%%%%%%%%%%%%%%%%%%%%%%%%%%%%%%%%%%%%%%%%%%%%%%%%%%%%
\tableofcontents
\newpage
%%%%%%%%%%%%%%%%%%%%%%%%%%%%%%%%%%%%%%%%%%%%%%%%%%%%%%%%%%%%%%%%%%%%%%%%%%%%%%%%%%%%%%%%%

\chapter*{Introduction}
\addcontentsline{toc}{part}{Introduction}

De multiple réseaux, y compris les réseaux sociaux, les réseaux informatiques, se divisent plus ou moins naturellement en communautés. La détection de cette structure sous-jacente aux réseaux constitue un problème actuel, et de nombreuses approches ont été développées pour y répondre. 

Dans ce rapport nous allons présenter une approche communément utilisée en apprentissage non supervisé, permettant de quantifier de la validité d'un partitionnement du réseau, les déffaillances à cette approche et la mise en pratique des méthodes utilisées pour y répondre. 

\section*{Concept de Modularité}
L'étude d'éventuelle structures communautaires dans des réseaux peut formellement être présentée par l'étude de graphe. Ainsi nous considérons un réseau comme un graphe, et emmettons certaines hypothèses à notre étude : 

\begin{center}
$Soit~G = (V,E)$ 
\end{center}



\end{document}